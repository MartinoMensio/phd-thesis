% \digraph{abc}{
%   rankdir=LR;
%   a -> b -> c;
% }


% \digraph{structs} {
%     node [shape=record];
%      rankdir=LR
%     struct1 [label="<f0> left|<f1> mid dle|<f2> right"];
%     struct2 [label="<f0> one|<f1> two"];
%     struct3 [label="hello\gvnewline world |{ b |{c|<here> d|e}| f}| g | h"];
%     struct1:f1 -> struct2:f0;
%     struct1:f2 -> struct3:here;
% }

% \resizebox{\textwidth}{!}{
\digraph{chap6} {
    rankdir="LR";
    node [shape=record];
    annotation [label="annotations"];
    a1 [label="topic"];
    a2 [label="leaning"];
    a3 [label="propaganda"];
    % combination [label="combination"]
    e_type1 [label="refine topic methodology\gvnewline (Sec. 6.3)"]
    e_type2 [label="answer RQs"]
    e1 [label="topic at different granularities\gvnewline (Sec. 6.3.1)"]
    e2 [label="topic across leanings\gvnewline (Sec. 6.3.2)"]
    e3 [label="propaganda across topics\gvnewline RQ1 (Sec. 6.4)"]
    e4 [label="propaganda across topics+leanings\gvnewline RQ2 (Sec. 6.5)"]
    e5 [label="CLF (topic, propaganda)  leaning \gvnewline RQ3 (Sec. 6.6)"]
    
    annotation -> a1;
    annotation -> a2;
    annotation -> a3;
    % a1 -> combination;
    a1 -> e_type1;
    a1 -> e_type2;
    % a2 -> combination;
    a2 -> e_type1;
    a2 -> e_type2;
    a3 -> e_type2;
    % combination -> e_type1;
    % combination -> e_type2;
    e_type1 -> e1;
    e_type1 -> e2;
    e_type2 -> e3;
    e_type2 -> e4;
    e_type2 -> e5;
}
% }