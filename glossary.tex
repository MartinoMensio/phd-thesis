\label{chap:glossary}

\newacronym{kmi}{KMi}{Knowledge Media Institute}

\newacronym{rq}{RQ}{Research Question}

\newacronym{ml}{ML}{Machine Learning}

\newacronym{nlp}{NLP}{Natural Language Processing}

\newacronym{use}{USE}{Universal Sentence Encoder~\citep{cer2018universal}}

\newacronym{bert}{BERT}{Bidirectional Encoder Representations from Transformers~\citep{devlin2018bert}}

\newacronym{lda}{LDA}{Latent Dirichlet Allocation~\citep{blei2003latent}}

\newacronym{hac}{HAC}{Hierarchical Agglomerative Clustering}

\newacronym{ari}{ARI}{Adjusted Rand Index~\citep{rand1971objective}}

\newacronym{ami}{AMI}{Adjusted Mutual Information~\citep{vinh2009information}}

\newglossaryentry{topic}
{
        name=topic,
        description={\todoAW{Not sure what this is adding. What's the difference between topics and events?}(wide): topics can be defined in a hierarchy, where top level are very wide (e.g. politics, tech, health, ...).
        (narrow): a set of events that are related, for example that originated from a seminal\todoAW{? What does this mean?} event}
}
\newglossaryentry{source}
{
        name=source,
        description={A source, or more specifically intended in this work as a \emph{news source} or \emph{news outlet}, is anything that provides news information for a period of time. Examples: BBC, The Guardian, Breitbart}
}
\newglossaryentry{event}
{
        name=event,
        description={anything that happens, especially something important or unusual (Cambridge Dictionary)}
}
\newglossaryentry{seminal-event}
{
        name=seminal event,
        description={an event that is at the start of a set of other events, linked by temporal dependency or causal dependency}
}
\newglossaryentry{story}
{
        name=story,
        description={a news article, that focuses on a main event but also contains references to other events}
}
% \newglossaryentry{parallel-news}{
%         name=parallel news,
%         description={a group of news articles that are published in a unique\todoAW{Not sure what you mean by "unique timeframe here". Is this different from "published at the same time"? (Is there a term used in media studies for what you're after here? A publication date or something?)} timeframe (e.g., same day) and cover the same events.\todoAW{So does that mean that two news articles that are published a month apart but describe the same events, would not be parallel?} They may use similar or different wording and may select different details to report. But they share the underlying truth.\todoAW{Isn't this already covered by your comment that they "cover the same events"?}}
% }
\newglossaryentry{corroborate}
{
        name=corroborate,
        description={Containing a piece of specific information which is confirmed by other sources. As more sources present the same information, even with slightly different terms, the information shared is \emph{corroborated}}
}
\newglossaryentry{omit}
{
        name=omit,
        description={Not including a piece of specific information, which instead appears in other sources}
}
\newglossaryentry{clique}
{
        name=clique,
        description={The original definition is that of a complete subgraph: every pair of nodes belonging to a clique has an edge that links the two nodes~\citep{luce1949method}. In this work, we use the term \emph{document-level clique} and \emph{sentence-level clique} together with a similarity threshold to indicate a group of documents or sentences that are all similar enough the one to the others. It differs from \emph{clustering} because single nodes do not need to belong to a cluster, they can be on their own}
}
\newglossaryentry{narrative}
{
        name=narrative,
        description={a particular way of explaining or understanding events 
            https://dictionary.cambridge.org/dictionary/english/narrative
        }
}
\newglossaryentry{persuasion}
{
        name=persuasion,
        description={The act of influencing a person's beliefs, attitudes, intentions, motivations, or behaviours~\cite{gass2018persuasion}. We use this term to encompass several phenomena, such as sentiment, \gls{propaganda}, \gls{populism}, coercion}
}
\newglossaryentry{populism}
{
        name=populism,
        description={Populism is ``a type of politics that claims to represent the opinions and wishes of ordinary people"\footnote{\url{https://www.oxfordlearnersdictionaries.com/definition/english/populism}}}
}
\newglossaryentry{sentiment}
{
        name=sentiment,
        description={A view or opinion that is held or expressed. Here in this work, we use two main axes: polarity, which represents how positive or negative the sentiment is, and strength, which instead represents how strong the opinion is expressed through the text}
}
\newglossaryentry{propaganda}
{
        name=propaganda,
        description={Use of language that has the specific goal to persuade the reader. It has many points in common with argumentation/rhetorics and is usually associated with deceiving techniques and partial points of view.
        %\todoAW{Very vague. Does this accurately characterise how you use the term throughout the dissertation?}
        It is a specific form of persuasion
        %\todoAW{I'm not sure that this use of persuasion matches what you've written in the previous definition.}
        that has the goal to indoctrinate
        %\todoAW{So propaganda is specifically about indoctrination?}
        population towards an individual or a particular agenda}
}
\newglossaryentry{propaganda-technique}
{
        name=propaganda technique,
        description={One of the practical ways that propaganda uses in language. Many of them in the literature, overlap with logical fallacies. Reference in this work comes from \cite{TODO}, 18 techniques in computational detection}
}
\newglossaryentry{political-leaning}
{
        name=political leaning,
        description={From far left to far right, passing through the centre. In this work it is taken as a mono-dimensional simplification of political viewpoints}
}
