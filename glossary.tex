\newacronym{kmi}{KMi}{Knowledge Media Institute}

\newacronym{use}{USE}{Universal Sentence Encoder~\citep{cer2018universal}}

\newacronym{bert}{BERT}{Bidirectional Encoder Representations from Transformers~\citep{devlin2018bert}}

\newglossaryentry{topic}
{
        name=topic,
        description={(wide): topics can be defined in a hierarchy, where top level are very wide (e.g. politics, tech, health, ...).
        (narrow): a set of events that are related, for example that originated from a seminal event}
}
\newglossaryentry{source}
{
        name=source,
        description={A source, or more specifically intended in this work as a \emph{news source} or \emph{news outlet}, is ``anything that provides news information for a period of time is said to be a news source''}
}
\newglossaryentry{event}
{
        name=event,
        description={something that happened in a specific time and place (definition from TDT: http://www.cs.cmu.edu/~nmramesh/p425-nallapati.pdf}
}
\newglossaryentry{seminal-event}
{
        name=seminal event,
        description={an event that is at the start of a set of other events, linked by temporal dependency or causal dependency}
}
\newglossaryentry{story}
{
        name=story,
        description={a news article, that focuses on a main event but also contains references to other events}
}
\newglossaryentry{corroborate}
{
        name=corroborate,
        description={Containing a piece of specific information which is confirmed by other sources. As more sources present the same information, even with slightly different terms, the information shared is \emph{corroborated}}
}
\newglossaryentry{omit}
{
        name=omit,
        description={Not including a piece of specific information, which instead appears in other sources}
}
\newglossaryentry{clique}
{
        name=clique,
        description={The original definition is that of a complete subgraph: every pair of nodes belonging to a clique has an edge that links the two nodes~\citep{luce1949method}. In this work, we use the term \emph{document-level clique} and \emph{sentence-level clique} together with a similarity threshold to indicate a group of documents or sentences that are all similar enough the one to the others. It differs from \emph{clustering} because single nodes do not need to belong to a cluster, they can be on their own.}
}
\newglossaryentry{narrative}
{
        name=narrative,
        description={a particular way of explaining or understanding events 
            https://dictionary.cambridge.org/dictionary/english/narrative
        }
}
\newglossaryentry{propaganda}
{
        name=propaganda,
        description={Use of language that has the specific goal to persuade the reader. It has many points in common with argumentation/rhetorics and is usually associated with deceiving techniques, partial point of views}
}
\newglossaryentry{propaganda-technique}
{
        name=propaganda technique,
        description={One of the practical ways that propaganda uses in language. Many of them in the literature, overlap with logical fallacies. Reference in this work comes from \cite{TODO}, 18 techniques in computational detection}
}
\newglossaryentry{political-leaning}
{
        name=political leaning,
        description={From far left to far right, passing through the centre. In this work it is taken as a wide simplification of political viewpoints}
}
