\label{chap:linguistic_persuasion}

Linguistic proxies of persuasion (different means that are used by articles to influence the reader)

Motivation: analyse if terms that change between multiple are correlated with linguistic proxies of propaganda


RQ: Is there a link between the parts that are different and propaganda/loaded language? 


strong/loaded language: using sentiment analysis tools 
18 techniques of propaganda


Experiment together with common ground search
Findings: differences in text != propaganda language
Different parts are not always loaded with loaded language or propaganda
A lot of changes are not meaningful in terms of propaganda: linguistic variance


\section{Propaganda variations across similar articles}

sentiment/propaganda across related articles

Why
What
How
Results

\section{Removing propaganda to make better similarity?}

effect of sentiment/propaganda words on sentence clustering

Why
What
How
Results

\section{Propaganda vs Populism}

propaganda vs populism

Why
What
How
Results

- dataset
