\label{chap:linguistic_persuasion}

In this chapter, we add our second ingredient: persuasion (in general) / propaganda (in specific).

Linguistic proxies of persuasion (different means that are used by articles to influence the reader)

Motivation: analyse if terms that change between multiple are correlated with linguistic proxies of propaganda


RQ: 
\begin{enumerate}
    \item What do writers use to persuade the reader?
    \item Is there a link between the parts that are different and propaganda/loaded language?
\end{enumerate}



strong/loaded language: using sentiment analysis tools 
18 techniques of propaganda


Experiment together with common ground search
Findings: differences in text != propaganda language
Different parts are not always loaded with loaded language or propaganda
A lot of changes are not meaningful in terms of propaganda: linguistic variance


\section{Related articles}

\subsection{Propaganda variations across similar articles}

sentiment/propaganda across related articles

Why
What
How
Results


Sentiment

% % sentiment
% Instead for the sentiment analysis, since we want to have detailed information (e.g. which specific words contain sentiment, and with which properties), we are relying on lexicon-based tools. Other more advanced tools (e.g. Stanford CoreNLP) have models which do not provide fine-grained scores but only sentence/document level. (This could be improved)
% % which sentiment lexicons?
% We selected some lexicons: Sentistrength, Vader, and AFINN (TODO description).
% % problems?
% The problem of doing sentiment analysis in this way is that the lexicon is recognised without accounting for other constraints (e.g. POS): we needed to remove some tools because they detected the word "Trump" as being positively loaded (trump as trumpet instead of Donald Trump).

\subsection{Removing propaganda to make better similarity?}

effect of sentiment/propaganda words on sentence clustering

Why
What
How
Results

\section{Propaganda vs Populism}

propaganda vs populism

Why
What
How
Results

Studying how much propaganda and populism are (correlation analysis)

- dataset
