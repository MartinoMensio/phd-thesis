\section{Introduction}

% TODO: Research Question of this chapter
% Summary of findings / contributions
% - limitations of current methods for corroboration/omission detection
% - similarity methods ? 
% - 

% chapter motivation: multiple versions of the same story 
% This chapter contains the first set of practical experiments that we carried out.

% general goal
The goal of this chapter is to analyse how different articles present the same event from different perspectives. For each event happening, there is a big multitude of articles which can be more or less similar between them.
And this variety of linguistical forms is an excellent opportunity to study the relationship between the bare facts (common ground) and the additional set of words / layering that try to persuade the reader of a certain opinion (with phenomena such as framing, interpretation, propaganda).

% fit into thesis
But how does this fit into the overall goal of the thesis? Our overall research questions aim at studying the relationship between propaganda and political leaning (overall RQ1: \textbf{What is the relationship between propaganda and political leaning?}).
% But, first of all, we need to take awareness that 
In order to study propaganda as the additional layer on top of information
(as discussed in Section~\ref{ssec:lit_layers_of_info})
we want to exploit the multitude of writings about the same topics. For this reason, this first practical chapter is about parallel news and comparing the shared information between the articles.
Given that the articles are a mix between raw facts and this additional layer of persuasion/opinion/propaganda, if we take into consideration many of them coming from different points of view (political sides/sources) we can then extract which part is shared and therefore is likely to be more closely related to the base layer of facts. 
%(TODO: to retake this afterwards and talk about credibility signals from bountouridis corroboration omission).
% Having this first differentiation between shared, omitted and unique, we also want to assess whether this is correlated to language that wants to persuade (propaganda). --> chapter 4 

Our first experiments, therefore, are directed at automatically comparing and finding differences in similar articles, and to try to understand the reasons for articles to be different or similar.

% RQ of this chapter
For this reason, we dive into the work of spotting similarities and differences in multiple articles.
% The overall RQ: \textbf{What is the relationship between propaganda and political leaning?}
The Research Questions that we address in this chapter are: 
\begin{enumerate}
    \item How do different sources present the same events?
    \item Can we analyse what is unique from each version and what overlaps?
    \item What type of document/sentence encoding performs better to detect corroborations and omissions? OR What are the characteristics that allow to encode better documents and sentences?
    % Is there a link between the parts that are different and propaganda/loaded language? --> chapter 4
\end{enumerate}




% find the shared information across several articles, and identify which parts are changed/unique. How sentences are changed between multiple articles.
% We take from the work of~\cite{bountouridis} and expand ...

% overview of subsections
In the following sections, we are starting in Section~\ref{sec:cgs_cross_referencing} by reproducing an experiment from a paper~\citep{bountouridis2018explaining} that automatically extracts corroborated and omitted information in groups of articles. Then in Section~\ref{sec:cgs_similarity} we explore more the methods to compute document similarity, then in Section~\ref{sec:cgs_clustering_and_differences} how we extract differences. Finally in Section~\ref{sec:cgs_findings} we discuss our findings.

% % method
% Method: 
% hierarchical clustering of sentences, omissions/corroborations + changed parts
% Ranking of similarity by models and by me


% Findings?
% threshold of similarity challenging to find. Meaning and sentence similarity don’t always go together (all models tried)


% Limitations?



% Outline

% - cross-referencing articles (bountouridis)
% - similarity
% - sentence clustering and small differences





% \section{from upgrade report (TODO)}

% During the first year, several activities have been done.
% Some have been done as initial explorations in the field of research, understanding what other researchers have done, ``getting the hands dirty'' with data and NLP tools.
% And their function within the PhD project has been to lead to the Research Questions that we described earlier.
% Some other activities have been done to kick start the future experimentation, for example doing data collection of news articles or by beginning to implement some of the stages of the processing pipeline that will be used.
% % , with the goal to find and formulate proper Research Questions and to prepare the execution of the proposal.


\section{Corroboration and omission extraction}
\label{sec:cgs_cross_referencing}
% Experiment 1
% what
Articles share information, so the first step for our analysis is to identify which parts are in common and which others change.
We started by taking as reference the paper from~\citet{bountouridis2018explaining} which presents a methodology to analyse how much information overlaps between different similar documents, identifying points of information that are \gls{corroborate}[d] or \gls{omit}[ted].

The main message of the paper is that sources that \gls{corroborate} the most are also sources that get a higher ``factual reporting'' score (as measured by Media Bias/Fact Check).\footnote{\url{https://mediabiasfactcheck.com/}}
Instead, the sources that \gls{omit} the most are the ones which get a lower score.
% why
% We wanted to analyse this resource because it seems to be going in the broad direction of our project, analysing different presentations in the news of the same event.


% how
The paper has been analysed and reproduced to get a deep understanding of how it works. The implementation started with the code publicly provided by the authors,\footnote{\url{https://github.com/dbountouridis/InCredible}} but its incompleteness in some stages of the processing (e.g., the creation of the article-level cliques, and all the specific hyperparameters of the algorithms used) required integrating the codebase.\footnote{\url{https://github.com/MartinoMensio/InCredible}}
An example of the reproduced analysis can be seen in Figure~\ref{fig:cluster_similar_sentences_amiri}.

\begin{figure}[!htbp]
    \centering
    \includegraphics[width=\linewidth]{figures/cluster_similar_sentences_amiri.png}
    \caption{A sentence from a Vox article which is corroborated with two sentences coming from Fox News and Washington Post.}
    \label{fig:cluster_similar_sentences_amiri}
\end{figure}

% limitations
With the experiments reproduced, we have been able to inspect the cliques of documents and sentences identified by the model, seeing the following limitations:

\begin{itemize}
    % \item Their demo\footnote{\url{http://fairnews.ewi.tudelft.nl/InCredible/}} just shows one specific article as main and one specific clique (not very interesting)
    \item The \emph{document clustering} sometimes splits similar articles over different clusters, or in some cases has different stories that talk about a different detail within the same cluster (e.g., when a news story re-emerges because further details are discovered).
    This can be a consequence of having TF-IDF as the underlying method to represent the documents.
    This method is fast and efficient for coarse topic detection because it is based on bag-of-words which works well with specific terms that distinguish the topics.
    But when we need to have a finer-grained clustering such as in this case, the limitation of this method may surface because the terms of two political events with the same entities mentioned result in having similar feature vectors.
    It is not enough to change the thresholds to obtain better document clusters.
    \item The \emph{sentence clustering} method provided just says that two sentences are similar, but does not point to which specific words are responsible for the similarities and differences. This would require a fine-grained analysis that in this paper is not included. Furthermore, sentences in a clique are very similar, and no significant differences have been observed because the similarity metrics are based again on TF-IDF. This method is not robust enough to the usage of synonyms and other variations on the linguistic surface, while at the same time is unable to distinguish two sentences that use the same words but have different meanings because of the sentence structure or of the role of the words, so it makes selecting a threshold value very difficult.
    \item The \emph{clique algorithms} are not the best choice for doing clustering when we have the information of how much similar two items are (a real-value instead of a binary-value is available from the similarity metric). The approach considers an unweighted version of the similarity graph by using a threshold (weights are just used to select the most appropriate clique during their creation), but instead dealing with the original weighted graph would allow better clustering techniques. %, like agglomerative clustering
\end{itemize}


\begin{figure}[!htbp]
    \centering
    \includegraphics[width=\linewidth]{figures/bountouridis_fig_3_noGA.pdf}
    \caption{Reproduced figure showing correlation between corroboration and credibility score.}
    \label{fig:figure_3_bountouridis_reproduced}
\end{figure}


% In addition to these problems, the model described is based on TF-IDF which is not as robust with changes on the linguistic surface (as we saw in the next experiment).
% Models flourished
% This is the motivation for 5.1.2

% And also, it uses the similarity between TF-IDF just with a threshold, modelling the graph and cliques as unweighted (the weights are just used to select the most appropriate clique during their creation).

% role of this experiment
This experiment helped to see the limitation of this type of work. %, which belongs to the \emph{similarity} area of research (Section~\ref{sec:lit_relationships}).
This paper provides a great way to analyse the overlap between articles and extracts pieces that have been omitted or that are corroborated, but does not investigate further the reason behind the selection of what to include or not.
This opened up for further work to:
\begin{enumerate}
    \item investigate how to represent the documents better to provide meaningful similarity metrics (Section~\ref{sec:cgs_similarity});
    \item experiment with document and sentence clustering to bring up differences (Section~\ref{sec:cgs_clustering_and_differences});
    % \item investigate the works that analyse framing theories and detection;
    \item investigate the reason for such differences to exist. As we will do in the next Chapter~\ref{chap:linguistic_persuasion}
    % \item collect data from more recent articles that would be more relevant and interesting.
\end{enumerate}
% \emph{i)} , \emph{ii)} , \emph{iii)} , and \emph{iv)} g.

% Apart from these limitations, we are building our processing pipeline on top of this type of analysis, that links together news articles at different granularity levels (documents, sentences, words).
% This gives us the opportunity to use features from multiple articles for the next stage of automated detection of framing techniques.
This method of comparison, has its main limitations in the similarity metric chosen. So we opted for doing some further research on the similarity metrics, in a way that accounts more for semantic similarity instead of just being term-based.


\section{Models for similarity analysis}
\label{sec:cgs_similarity}
% Experiment 2
% what
Moving to the problem of representing documents and sentences in a way that captures more semantic similarity, we decided to analyse closer the existing works, including word embeddings and language models.
We wanted to see in practice how the usage of different representation models would affect the measurements of similarity, experimenting with a small set of articles. 
% finding and exploring more advanced methods to find the similarity between texts by using language models, we experimented on how to use these methods.
% why?
Having a solid base for computing the distances between articles and sentences is a pillar for comparing different articles. The applications of similarity range from document clustering to the identification of omitted pieces of information in a cluster, therefore it is very important to use a proper method that is not deceived by the usage of synonyms and other linguistic variations in communicating the same information. To study the differences in the language of framing, we first need to be able to tell whether two pieces of text are discussing the same information, and distinguish degrees of similarity properly.

To study the similarity metrics, we use two different experiments. The first, detailed in Subsection~\ref{ssec:cgs_similarity_qualitative}, where we observe qualitatively the effects of using one similarity metric or others when comparing sentences. The second, in Subsection~\ref{ssec:cgs_similarity_cliques}, instead observes the effect of such choice on the downstream task of corroboration and omission extraction.
We conclude with some observations in Subsection~\ref{ssec:cgs_similarity_conclusion}.


\subsection{Qualitative differences benchmark}
\label{ssec:cgs_similarity_qualitative}
% how?
% Specifically on the sentence level, we experimented to see how different models were able to pick similar sentences, by setting up a small benchmark.
We set up a small benchmark where the goal is to find the most similar pairs of sentences coming from selected pairs of news articles which cover the same event. Each model candidate has to tell which 10 most similar pairs of sentences has found, one from one article and one from the other.
The pairs of articles have been chosen manually, by considering three constraints: \textit{i)} description of the same event, \textit{ii)} from different news outlets, \textit{iii)} published near in time, with a maximum time distance of one day.
Each model would extract the most similar 10 pairs, and we then compare the pairs provided and their relative order in the rankings.

The selected models used in the benchmark are the following:
\begin{itemize}
    \item \textbf{TF-IDF}: with a feature size of 2000, with a preprocessing made of lowercasing and tokenizing, without lemmatising;
    \item \textbf{GloVe-average}: considering GloVe word embeddings trained on the CommonCrawl dataset, and doing an average of the vectors over the sentence;\footnote{\url{https://spacy.io/models/en\#en_core_web_lg}}
    \item \textbf{BERT}: using the most popular embeddings provided by Google Research~\citep{devlin2018bert} with the base uncased pre-trained weights;\footnote{\url{https://spacy.io/models/en-starters\#en_trf_bertbaseuncased_lg}}
    \item \textbf{USE}: using sentence embeddings coming from Universal Sentence Encoder~\citep{cer2018universal} which has been specifically trained for sentence similarity.\footnote{\url{https://tfhub.dev/google/universal-sentence-encoder/}}
\end{itemize}

In all the cases, the representations from these models have been compared with the cosine similarity.
For each pair of sentences that was provided by any of the models, we listed by manual analysis which differences were contained, in terms of details that changed, or different words used.

\todo{Table with example sentences and and ranking showing USE is better}


\begin{figure}[!htb]
    \centering
    \includegraphics[width=0.9\linewidth]{figures/lyra.pdf}
    \caption{The comparison between two sentences, one from the BBC and the other from Sky News, where multiple differences exist.}
    \label{fig:lyra}
\end{figure}

An example can be seen in Figure~\ref{fig:lyra} that shows a sentence from the BBC\footnote{\url{https://www.bbc.co.uk/news/uk-england-hereford-worcester-51791346}} and one from Sky News,\footnote{\url{https://www.dailymail.co.uk/news/article-8088805/Britons-facing-heavy-downpours-four-inches-rain-50mph-winds-set-batter-UK.html}} where the differences are the following:

\begin{itemize}
    \item the detail ``4x4'' just appears on the BBC article;
    \item the detail ``on the night of 18 April 2019'' just appears on the BBC article;
    \item Active vs passive sentence ``a masked gunman fired'' vs ``she was hit by a bullet fired by a masked gunman'';
    \item the detail ``a bullet'' just appears in the Sky article;
    \item ``Towards officers and onlookers'' vs ``towards officers'': in the second, the targets of the gunman are just the officers.
\end{itemize}

With this kind of information on the number, type and magnitude of changes contained in different pairs of sentences, we can get a qualitative idea of how the measures of similarity coming from the different models are representative of the effective differences. If a model scores more similar a pair of sentences that appear to us to be less related than another pair, that is a negative sign for that model. 


% Observations
% \todo{rewrite better the observations}
The observations that we have for the TF-IDF model is that the feature size changes the results a lot.
It requires to be computed on a set of documents all together (which also changes which features are selected), and it is not possible to encode an additional document without changing the representation of the already encoded documents.
We also see that the type of pre-processing affects the results: without a lemmatization step to the pipeline, it is sufficient to change the verb tense to have a different term.
Given these limitations, we find that sentences that are very similar in the meaning but have some differences in the linguistic surface see a drop in their similarity with this model.

Instead considering GloVe-average, we observe in some cases that the measure of similarity provided does not capture substantial changes in the meaning. The problem is that, while it can use wordwise similarity quite well, the sentence structure is not accounted for its representation. The representation is a simple average of the word vectors (e.g. ``Luke insulted John'' results in being equal to ``John insulted Luke'').

For the language models (BERT and USE models), we see a big improvement in the pairs of sentences that come as more similar.
% The values provided are very similar. This per-se is not a problem if some geometric properties are valid (ordering, proportions)
We observe that USE provides values less skewed to the higher end, distributing the similarity values more evenly. The numbers make more sense without any re-scaling technique, and therefore the heatmaps shown in this document come from this model.
It is also the only model considered that is purposely trained on a semantic similarity task, while the other models can provide similarity measures just because of how they represent language.


% \todo{an example of two pairs where we can see some of the limitations?}

\subsection{Application to corroboration and omission extraction}
\label{ssec:cgs_similarity_cliques}
% Cliques with USE

In this section we take the experiment from Section~\ref{sec:cgs_cross_referencing} and substitute the TF-IDF encoding method with a semantic similarity model (USE).

While in the previous Subsection~\ref{ssec:cgs_similarity_qualitative} we were assessing the quality of USE vs TF-IDF on their own, here we see the difference when applying them to downstream tasks.

\todo{Show figure with USE, that results are better correlated? NO}

In figure XXX, we can see that when we compare the base model TF-IDF with USE, we get a better correlation between corroboration and credibility.
OR: we get unclear results. However, when inspecting the cliques, we see that 

\todo{need an example}

Show cliques inspection quality


\subsection{Conclusion of section}
\label{ssec:cgs_similarity_conclusion}

% role of this experiment
This experiment shows the need for a similarity model that accounts for the semantics more than the linguistic surface. And given the continuous progress of language models, we need to be able to switch our choice relatively easily.
For example, by looking at the Semantic Textual Similarity benchmark,\footnote{\url{http://nlpprogress.com/english/semantic_textual_similarity.html}} at the moment the best model available is XLNet~\citep{yang2019xlnet} but this could change at any time.

For this reason, ????
%so we will use it for our future experiments.
% this means for us:
% - we need to use a similarity resistant to changes in the linguistic surface
% - we need a measure that is able to represent well the different levels of similarity
% - we must be able to switch the model used easily, in case new public benchmarks for STS show a different winner (example XLNet~\cite{yang2019xlnet}).

%The purpose of this experiment is to have a good observation of how different types of models can be effective or not, and to experiment with them to drive the implementation of the processing pipeline.
% Benchmark, availability of code and maybe further measures on our system will decide the final ``winner''.
% Purpose: implementation and building of the pipeline.


\section{Sentence clustering and extraction of differences}
\label{sec:cgs_clustering_and_differences}
\todo{change title of section to something like 
1) Fine-grained differences extraction
2) Spotting changed terms}
% Experiment 3
% what
This section has two goals: 1) to improve the cliquing approach used in Section~\ref{sec:cgs_cross_referencing} and 2) to take a closer look at the fine-grained differences between highly-similar sentences and study the uniqueness of the words used.

% The next experimentation that we have done regards the usage of the similarity values to group together sentences describing the same details and at the same time study the uniqueness of the words used.

\subsection{Hierarchical sentence clustering}
\label{sec:cgs_clustering_and_differences_hierarchical}


% why
We have seen with the reproduction of the model from \citet{bountouridis2018explaining} that one big limitation of using cliquing techniques over unweighted graphs is that they do not exploit the full power of the distances available, which resulted in having fragmented clusters due to a choice of the ``similar-enough threshold'' that is very sensible.
% We have also experimented with different embedding models and we want to use them
% \todo{from here on}
% This comes from the limitation of the first experiment of reproduction of the paper. (from experiment 1)
% (from experiment on similarity)

% how
With this idea, we retrieved some groups of articles that relate to the same event from Google Headlines, which aggregates and clusters together news articles from multiple sources.\footnote{\url{https://www.blog.google/products/news/new-google-news-ai-meets-human-intelligence/}}
These documents are processed with the SpaCy NLP Python library\footnote{\url{https://spacy.io/}} to split the documents into sentences and have available different NLP functions (e.g., tokenisation, POS tagging).

% 1. distance computation
Each of the sentences is then passed through a language model which creates a sentence embedding, in this case using the Universal Sentence Encoder because it showed to distribute the similarity values more evenly and is specifically trained for sentence similarity.

% 2. hierarchical clustering (example with diagram)
We then use agglomerative hierarchical clustering for different reasons:
\begin{itemize}
    \item it does not require the specification of the number of clusters wanted, we want to be flexible;
    \item we can truncate the clustering when we reach a certain level of distance between the clusters, or a certain number of clusters;
    \item We can see the evolution of many different features (e.g., number of clusters, size, internal cohesion) while performing the clustering step by step;
    \item we have a graphical representation (dendrogram) which helps to inspect and understand what is happening;
    \item it has widely been used for similar tasks (e.g., finding related claims~\citep{almeida2020text})
\end{itemize}

% This clustering algorithm has the following parameters:
% \begin{itemize}
%     \item linkage method: how to choose which clusters to merge. Different strategies exist: Ward: minimise the total within-cluster variance (weighted squared distance between cluster centres). Single: Nearest Point Algorithm. Complete: Farthest Point Algorithm
%     \item distance function: cosine, euclidean, ...
% \end{itemize}
% \todo{describe why ward and cosine look better}

As we can see in Figure~\ref{fig:dendrogram}, we can explore what is the distance required to have different sentences inside the same cluster, and select a certain threshold more consistently.\todo{bigger figure}
\begin{figure}[!htb]
    \centering
    \includegraphics[width=\linewidth]{figures/dendrogram.png}
    \caption{A portion of the dendrogram that shows how different sentences are merged in the clusters by increasing distance values.}
    \label{fig:dendrogram}
\end{figure}

Using this type of inspection, we can decide specific threshold values more consistently: we can see what is the required similarity to make two pairs of sentences be in the same cluster.
% \todo{some observations about the distance values and threshold}

% 3. extract degree of uniqueness of words (from pairwise to clusterwise, with bag-of-words or difftool (order matters, duplicates))
% second motivation: highlight the different words and their uniqueness
With this method, we create sentence clusters that are very similar in their semantic content, but at the same time have linguistic changes. This is a joined effect of having models that deliver a better similarity metric and also of applying a weighted-similarity approach when building the clusters and not just a binary approach.

\todo{conclusion from this hierarchical analysis?}

Selected hyperparameter with USE by using the dendrogram and by looking at the samples:
- threshold distance=0.6 euclidean distance ward

From this experiment, we learn that ???
- how do we know this is better than cliques? Ground truth of clusters is not available. A benchmark test would be required to claim superiority.

\subsection{Term uniqueness}
\label{sec:cgs_clustering_and_differences_uniquenesss}

\todo{more motivations than this?}

To facilitate an analysis of the differences, we experimented with different methods of highlighting the uniqueness of the words in a cluster.

Inspired by software engineering tools, we first tried with algorithms based on \texttt{diff}~\citep{myers1986ano}. Diff (also coming as UNIX \texttt{diff} tool) is a great resource for comparing different modified versions of the same document.
Suddenly we realise that one of the main advantages of diff analysis, the position, does not really matter a lot in our analysis, because natural language is more flexible and reordering phrases and sub phrases should not be accounted for in our analysis. So with the relaxed constraint on the position of the words, we moved to a measure of term uniqueness based merely on the words appearing or not in the considered sentences. For this reason, we named this comparison \texttt{set-based} instead of \texttt{diff-based} because we treat the words for each sentence as a set (unused position of words, repetitions not considered). Although we do not consider repeated words in a sentence as modifying the uniqueness, we think that repetition should not be considered when looking at single sentences and only be looked at when considering whole documents (repetition in an article may carry some persuasion).

We therefore define a scale of uniqueness in the following way:
$$u_w = 1 - \frac{|\set{s_i | s_i \in S \land w \in s_i}|}{|S|}$$
where S is the set of sentences considered, $w$ is the word for which to compute the index. The expression compares the sentences of the current cluster where $w$ appears (numerator) with respect to the cluster size.
A value close to $1$ means that the word is used in just a few sentences in the cluster.
This gives, in the example below, a higher value of uniqueness to the word ``deceased'' that just appears in one over three sentences ($u = 2/3$), while ``surgery'' has $u = 0$.

The application of this metric can be seen in Figure~\ref{fig:words_uniqueness} where we score each of the words with their uniqueness score.

\begin{figure}[!htb]
    \centering
    \includegraphics[width=\textwidth]{figures/words_uniqueness.png}
    \caption{A sentence cluster example, where the uniqueness of words in the cluster is highlighted. In this way, we can instantly see the words that are most unique.}
    \label{fig:words_uniqueness}
\end{figure}


% not only uniqueness, but how it was used to observe variations of terms
This uniqueness was used to inspect many sentence-level clusters that were gathered by using the hierarchical clustering as described in Section~\ref{sec:cgs_clustering_and_differences_hierarchical}. Helped by the uniqueness degree of the words, we are able to spot the differences between the sentences faster and classify the main types of singular words that change:
\begin{enumerate}
    \item verb tenses: not meaningful
    \item chunks of information at the beginning or end of the sentence. Usually, those pieces are just a matter of segmentation of the sentences (e.g. the reporter actually includes the omitted terms in the next or previous sentence) 
    \item details that are present in some articles and omitted in others (e.g. Figure~\ref{fig:words_uniqueness})
    \item verb choices that convey a slightly different message
\end{enumerate}

\subsection{Conclusion of section}
\todo{findings from this?}
We try with this experimentation on one side to group the sentence accordingly to the semantic similarity (more broad and resistant to changes in the linguistical surface) and from the other to be able to extract and see how the word-level details change.
It may look like a contradiction, but we need this double analysis to link different sentences that share the same stories and to be able to see the actual differences between them.

From our observation, we see that
there are many variations that are meaningless, some that instead are able to present things very differently from a few words. Similarity alone cannot distinguish between them.
Most of the differences observed are meaningless.



% role of this experiment and outcomes
% This methodology can be used in our framework to help the preparation of the dataset in different steps.
% First of all, during the creation of article couples that have to be compared (using at the article-level the same clustering methodology). In this case, we will need to use a specific threshold that cuts out unrelated articles but at the same time keeps a considerable number of differences on the document level (not too similar because identical articles, that are a lot, are not useful).
% While for the first user study we can exploit hand-curated groups of articles, when doing a larger creation of the dataset we will need to rely on automated techniques.
% This methodology could also be used as support for the annotators, to see both which sentences are most similar, and the words that differ inside. This would make the annotation process faster and easier.

% This experiment needs to be completed, to choose some parameters with better criteria. We want to identify good intervals for the thresholds and parameters (e.g. with euclidean distance around 0.6-1.0 sentences start to have linguistic variations but still very related to the same concepts).

% This experiment evidences the need to explore more on the interpretation of the differences, pushing for the user study of RQ1.1.

% It serves the RQ1.2 as a first implementation of the processing pipeline by making available different articles and their document and sentence-wise relationships. Building on top of these features, we can then develop the methodology for doing the cross-article framing analysis.




\section{Data used}
% what
% We also started collecting, early this year, several types of data that will be useful for the analysis planned.
For this first chapter, we have been focusing on parallel news dataset. We started by using existing datasets, and then we started collecting on our own because we wanted specific features.
% why
When looking for data, we are interested in different features.
First of all, a wide set of articles is needed, dense in time and from a wide variety of news outlets. We need substantial overlap between articles and the more sources we can include, the better we can observe variations of the framing phenomena.
Another very important feature is to have a good pre-clustered set of articles to help the curation of a dataset, %especially for the first Research Question, 
by knowing that the articles are well related.
Then when we will have the document clustering in action, this feature is not anymore required, but still can serve as a benchmark for that stage.
And another desirable feature would be to have articles that come from sources with a different opinion. %, that would be beneficial to create examples especially for the user study, where we want to maximise the occurrence of framing techniques. 
This is because the general goal of this thesis is to study the persuasion and propaganda, so if the points of view are different we can see the propaganda easier.

% how
% - google news (more than 500k articles) daily. Some stats about the number of sources involved  TODO: how is it made? What are its properties? Why useful?
Given these requirements and after exploring different news aggregators, we found that Google Headlines Full Coverage feature\footnote{\url{https://www.blog.google/products/news/new-google-news-ai-meets-human-intelligence/}} would fit the requirements of covering a big number of news sources (the \texttt{en-GB} version contains articles from more than 10k domains) and being very dense (an average of 9k new articles each day).
The data comes divided by topics (Latest, United Kingdom, World, Business, Technology, Entertainment, Sports, Science, Health) and inside each topic, the articles are grouped in ``stories''. Each story has articles from the most relevant sources (``Top coverage'') and then also lists articles from other less important sources, for an average of 21 articles inside each story.
The stories are created automatically by Google News and this allows it to be always updated and be so diverse in the sources included.
We have captured during the first year (mid-March to September) every day the published set of stories, and managed to retrieve more than 700k articles (as of the end of June, and not all the articles listed can be retrieved because of paywalls or other filtering techniques by the publishers).

% - allsides: human-created with interesting framing differences
Another data source that we actively retrieve is AllSides which provides a curated set of ``headlines''\footnote{\url{https://www.allsides.com/story/admin}} where three articles with a different political alignment are put together and compared in their difference.
The curators describe how the story gets framed by the considered sources, using natural language.
This description usually contains the usage of terms or themes that get mentioned.
%At the end of June, we have available 4764 headlines, with 13979 articles linked.
Differently from Google Headlines which has different versions for each country, this data is US-focused being curated in the US and therefore has a much more limited scope. Also, the discussion of bias and framing is mainly focused on political issues, while we want to focus also on other types of differences of opinion.
% role of this specific data
This data, although the description of the differences is not directly parseable, will be used to feed the user study and understand the role of comparing different sides.

% - allnews: standard benchmark, wide adopted (find refs)

% scraping is legal for research: https://aballatore.space/2020/04/01/web-scraping-is-legal/

% % role of this
% The data collection done in these current months will continue across the PhD, and will be used in different stages of the analysis. It will serve as the seed to create the labelled dataset for the first Research Question and also provide a wide set of articles from different news sources to empower the studies of the second Research Question.
% Having articles from so many different news sources, we can on one side provide some indication of framing for news sources that are not usually targeted by manual framing studies because not enough ``important'', and on the other side be more confident to observe some phenomenon of information re-usage that is the underlying hypothesis for the last sub-question.
This data was used for this and the second chapter. From Chapter~\ref{chap:chap_5} instead we rely mainly on a existing dataset, still coming from AllSides ~\citep{baly-dataset} because we want to compare our results with other works.

% \section{NO: Formalisation and dissemination}
% % what
% The last type of activity carried out during this year has been the presentation of this work to other researchers throughout different events, both internally to the Open University and externally.
% % why
% The motivation of this activity has been to get some feedback from both people working inside the same research space and also from other fields.
% From the first group, we wanted to get an expert opinion and mainly understand if we are missing some related research work that could be helpful.
% Instead, from the more general-audience group, we wanted to understand if this type of research makes sense to them and try to explain more motivationally and in an easier format.

% % how
% Belonging to the first group, this work has been presented firstly to an internal seminar in KMi on the 25th March, then to the Text2Story workshop part of the ECIR conference as a position paper which was presented in April\footnote{\url{http://text2story20.inesctec.pt/}}.
% This position paper~\cite{mensio2020towards} focuses on describing some proposed cross-article signals that would show differences in how stories are narrated.
% The proposal described completely in this report has also been presented in the CRC PhD Conference, that is an internal conference for PhD students of KMi and C\&C schools.

% For the more wide audience instead, during June we submitted a poster to the OU PhD Poster Competition which, involving a more general audience, focused more on being simple to understand.

% These documents can be found at the end of this file.


\section{Discussion}
\label{sec:cgs_findings}

From this chapter, we achieved to be able to understand the uniqueness of information at different granularities:
- when certain words just occur in one article (degree of uniqueness)
- when most of the articles report all the same things
- ?

\todo{I need something to link together this chapter and also need strong findings to conclude it.}


Possible findings:
\begin{enumerate}
    \item As already denoted by the work of~\citet{bountouridis2018explaining}, the first part of this analysis shows a positive correlation between corroboration and credibility of news outlets, and a negative correlation between omission and credibility. We went one step further, by being able to automatically find the specific words that change between multiple news articles, and identify the degree of uniqueness of them. We think that this is greatly important for many downstream tasks, such as showing to the user during annotation tasks or even when consuming news online. %: corroboration correlates positively to credibility of news outlets, omission negatively
    % \item Extreme left and right corroborate less and omit more? (but needs leaning? No if we just take centrality-extremism)
    % \item Delta Time of publication: more similar articles are published at similar times, instead the further the time delta the more further they are?
    % these are more limitations
    \item Observing similarities between multiple documents is made very difficult by linguistic variations. It is not only a matter of what is included and what is excluded. The similarity between sentences is also capturing some linguistic variations more than others (more about the surface and less about the intention).
    \item We do not have any interpretation of what a change in a document conveys. Something may be written differently just because of random choices, or there can be some hidden reasons and some goal to persuade (see more in next section and next chapter).
\end{enumerate}



\section{Next}
\label{sec:cgs_next}
% link to next chapter

This chapter gives us some insights into how similar documents are changed and modified across news sources, and how critical corroboration and omission are.

% recap contributions?


% open for new chapter
But we still don't have any idea about what these changes \textbf{convey}. It could be that the differences exist for different natures, originating from randomness (each author/source has different jargon, or by casualties).
Or on another side, there could be a purpose that is obscurely manifesting through these small choices. A purpose to influence the readers and subtly persuade/manipulate them.

What changes between the changed parts? What characterises the differences? / How can we characterise the differences quantitatively describing how they try to persuade the readers?

So for the next chapter, we want to investigate persuasive language and how it can be computationally quantified. 
