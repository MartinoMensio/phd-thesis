Test
\acrshort{kmi}

\section{Problem Statement}

\section{Motivation}

Motivation/Orientation: multiple narrations of the same events. Driven by multiple factors (selection criteria, point of view of author/publisher, relevance, agenda-setting, …)
This is linked with Persuasion, but more specifically propaganda. Communication with the goal to persuade. Subtle or more explicit.
What distinguishes one piece of news from the others about the same event? 

Political ideology → writer/editor → news (facts + propaganda(opinion)) → persuasion of reader

Rationale (niche): computational propaganda detection is at an early stage in the research community. Current detection is based on unbalanced datasets that particularly target right-leaning news.

Aims/Goals of this thesis: 
understand how propaganda varies across the political spectrum
Computational perspective: is propaganda detection ready to work in news independently from the political orientation/leaning of the source?

Hypotheses:
Propaganda language used by different political leanings is distinguishable
 Current propaganda detection is biased to recognise better right-leaning news


\section{Research Questions}

RQs:
What is the relationship between propaganda and political leaning?
Diffusion
Does propaganda exist all across political leaning?
Do existing propaganda datasets represent each leaning well?
(Why is there this discrepancy between i) and ii)?)
Term usage
What terms are used in propaganda?
Do different leanings use similar propaganda terms?
Populism
Is propaganda correlated to populism in literature?
Does the correlation show up in the data?
Topics (?)
How is propaganda spread across topics (overall)?
Do different leanings use propaganda with different topics?
Targets of propaganda (?)
Who are the targets of propaganda (overall)?
Do different leanings have different targets of propaganda?
Prediction
Is it possible to predict the political leaning of an article from its propaganda features?
By removing propaganda terms from articles? 


From the chapters (updated)
Chapter 3
\begin{enumerate}
    \item RQ1.1: How do different sources present the same events?
    \item RQ1.2: Can we analyse what is unique from each version and what overlaps? 
    \item RQ1.3: How can we automatically detect omission and corroboration across multiple articles?
    \item RQ1.4: How can we select better similarity metrics to empower omission and corroboration detection?
\end{enumerate}


Chapter 4
\begin{enumerate}
    \item RQ2.1: How could we automatically detect the persuasion techniques used by writers?
    \item RQ2.2: Which persuasion techniques do we detect most frequently? 
    \item RQ2.3: How are the differences between similar articles related to a different use of persuasion techniques?
    \item RQ2.4: How much of an obstacle is persuasion in recognising articles that are related to the same news?
\end{enumerate}


Chapter 5
\begin{enumerate}
    \item RQ3.1: How does persuasion vary across the political spectrum?
    \item RQ3.2: Can we predict the political leaning of a news article by observing the propaganda it uses? 
    \item RQ3.3: Is Propaganda Detection balanced? Or is there some imbalance in the datasets used in the literature?
\end{enumerate}



Chapter 6
\begin{enumerate}
    \item RQ4.1: How can we optimise the definition of topic in order to have enough details?
    \item RQ4.2 How propaganda changes across topics and leanings?
    \item RQ4.3 Knowing the topic, does it make classification (prop-->leaning) easier?
\end{enumerate}


\section{Research hypotheses}

\section{Research methodology}


\section{Contributions}

Contributions:
Identification of imbalance of the main propaganda datasets
Topics where current automated propaganda detection is more problematic (TODO) 
Link between propaganda and political leaning is weak (not enough to identify leaning by just looking at the propaganda techniques) → against HYP1
Imbalance is / is-not a problem for propaganda detection
Propaganda vs populism

\section{Structure of the dissertation}

Chapter 3: Common Ground Search
Chapter 4: Linguistic Proxies of Persuasion
Chapter 5: Perspectives and Political Sides
Chapter 6: Topics
Chapter 7: ?

\section{Publications}

\subsection{Related to this thesis}

Towards a Cross-article Narrative Comparison of News
M Mensio, H Alani, A Willis
Proceedings of the Text2Story’20 Workshop

\subsection{Other publications during the timeframe}

Mensio M., Alani H. (October 2019) News Source Credibility in the Eyes of Different Assessors. In Conference for Truth and Trust Online (TTO 2019), London, UK [Full text (ORO)] [Full text (Conference)] [Slides]

Mensio M., Alani H. (October 2019) MisinfoMe: Who’s Interacting with Misinformation? In (ISWC 2019) Posters and Demos [Poster] [Full text (ORO)] [Full text (CEUR)]


Mensio M., Bastianelli E., Tiddi I., Rizzo G. (March 2020) Mitigating Bias in Deep Nets with Knowledge Bases: the Case of Natural Language Understanding for Robots. In AAAI 2020 Spring Symposium on Combining Machine Learning with Knowledge Engineering [Full text (CEUR)]

Mensio M., Willis A., Alani H. (April 2020) Towards a Cross-article Narrative Comparison of News. In Third International Workshop on Narrative Extraction from Texts held in conjunction with the 42nd European Conference on Information Retrieval (Text2Story2020 @ ECIR2020) [Full text (ORO)] [Full text (CEUR)] [Presentation slides]

Mensio M., Alani H., Willis A. (June 2020) One Event, Different Stories. In Postgraduate Research Poster Competition 2020, The Open University [Multimedia entry (ORO)] [Poster (ORO)]

Burel G., Farrell T., Mensio M., Khare P., Alani H. (October 2020) Co-Spread of Misinformation and Fact-Checking Content during the Covid-19 Pandemic. In (SocInfo 2020) Social Informatics 2020 [Full text (ORO)] [Full text (Springer)]

Piccolo L., Blackwood A., Farrell T., Mensio M. (July 2021) Agents for Fighting Misinformation Spread on Twitter: Design Challenges. In 3rd Conference on Conversational User Interfaces (CUI 2021) [Full text (ORO)] [Full text (ACM)]

Denaux R., Mensio M., Gomez-Perez J., Alani H. (August 2021) Weaving a Semantic Web of Credibility Reviews for Explainable Misinformation Detection. In Thirtieth International Joint Conference on Artificial Intelligence (IJCAI-21) [Full text (ORO)] [Full text (IJCAI)]

Reyero Lobo P., Mensio M., Pavon-Perez A., Bayer V., Kwarteng J., Fernandez M., Daga E., Alani H. (June 2022) Estimating Ground Truth in a Low-labelled Data Regime: A Study of Racism Detection in Spanish. In (ICWSM-22) st Workshop on Novel Evaluation Approaches for Text Classification Systems on Social Media (NEATCLasS) [Full text (ICWSM)]
