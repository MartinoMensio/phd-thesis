\chapter{\statusorange Introduction}
\label{chap:intro}

% PhD Project done across \acrshort{kmi} and Computing\&Communications.

This chapter contains an introduction to the thesis.
 Section~\ref{sec:intro_problem} contains the Problem Statement, then Section~\ref{sec:intro_motivation} the Motivation, then Section~\ref{sec:intro_rqs} contains all of our Research Questions. Then in Section~\ref{sec:intro_hyp} we present our hypotheses, in Section~\ref{sec:intro_method} the general methodology used. We end the chapter with the main contributions in Section~\ref{sec:intro_contributions}, the structure of the dissertation in Section~\ref{sec:intro_structure} and the publication in Section~\ref{sec:intro_publications}.


\section{Problem Statement}
\label{sec:intro_problem}


At the intersection with misinformation, political issues, ...
Study on propaganda

What makes news articles about the same issue different? Details that are chosen, layers of framing/propaganda on top of the facts described.
What are the differences in persuasion/propaganda used by different political orientations?
How can we automatically recognise political leaning from propaganda?
What is the relationship with the topics discussed?

\subsection{Key concepts}

Parallel News

Similarity / Variations

Persuasion and Propaganda

Political Ideology and Political Leaning

Topic

For a complete list, we suggest to look at the Glossary (TODO ref)



\section{Motivation}
\label{sec:intro_motivation}

Who can find this thesis useful?
- Computational Research people:
    1) we underline the correlation between multiple variables (similarity / propaganda / leaning / topics) that may be helpful to classify better content (supported by the improvement of F1)
    2) We highlight the problem of imbalanced dataset for propaganda detection
- User reading the news: 
    1) support with a tool to highlight propaganda, link to other documents on how the same topic is covered
    2) understand better how propaganda distributes across leanings and topics, to be able to recognise more manipulation and be less manipulable.

Motivation/Orientation: multiple narrations of the same events. Driven by multiple factors (selection criteria, point of view of author/publisher, relevance, agenda-setting, …)
This is linked with Persuasion, but more specifically propaganda. Communication with the goal to persuade. Subtle or more explicit.
What distinguishes one piece of news from the others about the same event? 

Political ideology → writer/editor → news (facts + propaganda(opinion)) → persuasion of reader

Rationale (niche): computational propaganda detection is at an early stage in the research community. Current detection is based on unbalanced datasets that particularly target right-leaning news.

Aims/Goals of this thesis: 
understand how propaganda varies across the political spectrum
Computational perspective: is propaganda detection ready to work in news independently from the political orientation/leaning of the source?




\section{Research Questions}
\label{sec:intro_rqs}

RQs:

\subsection{RQ1: What makes news articles about the same issue different?}
/ How can we analyze and compare multiple sources to identify unique perspectives, overlapping information, detect omissions, corroborations, and select effective similarity metrics?
→ Chapter 3
\begin{enumerate}[label={\textbf{RQ1.\arabic*:}},leftmargin=2cm]
    \item How do different sources present the same events?
    \item Can we analyse what is unique from each version and what overlaps? 
    \item How can we automatically detect omission and corroboration across multiple articles?
    \item How can we select better similarity metrics to empower omission and corroboration detection?
\end{enumerate}

\subsection{RQ2: How can we automate the detection of persuasion techniques used by writers and assess their impact on recognizing related news articles?}
→ Chapter 4
\begin{enumerate}[label={\textbf{RQ2.\arabic*:}},leftmargin=2cm]
    \item How could we automatically detect the persuasion techniques used by writers?
    \item Which persuasion techniques do we detect most frequently?
    \item How are the differences between similar articles related to a different use of persuasion techniques?
    \item how much of an obstacle is persuasion in recognising articles that are related to the same news? (rewrite)
\end{enumerate}

\subsection{RQ3: How does political persuasion vary across the spectrum, and can we predict a news article's political leaning based on its propaganda techniques?}
→ Chapter 5
\begin{enumerate}[label={\textbf{RQ3.\arabic*:}},leftmargin=2cm]
    \item How does persuasion vary across the political spectrum?
    \item To what extent can we predict the political leaning of a news article by observing the propaganda it uses?
    \item Is Propaganda Detection balanced? Or is there some imbalance in the datasets used in the literature?
\end{enumerate}

\subsection{RQ4: How does the detected propaganda vary across topics and political leanings, and what is the impact of combining propaganda and topic features on accurately determining the political leaning of a news article?}
→ Chapter 6
\begin{enumerate}[label={\textbf{RQ4.\arabic*:}},leftmargin=2cm]
    \item How does the detected propaganda change across the topics? Are there major differences when considering “polarised” topics with respect to more neutral topics?
    \item How does the detected propaganda change across leanings in “polarised” topics? And how does it change in non-polarised ones?
    \item What are the effects of combining the propaganda features with the topic features, to recognise the leaning of a news article?
\end{enumerate}


% What is the relationship between propaganda and political leaning?
% Diffusion
% Does propaganda exist all across political leaning?
% Do existing propaganda datasets represent each leaning well?
% (Why is there this discrepancy between i) and ii)?)
% Term usage
% What terms are used in propaganda?
% Do different leanings use similar propaganda terms?
% Populism
% Is propaganda correlated to populism in literature?
% Does the correlation show up in the data?
% Topics (?)
% How is propaganda spread across topics (overall)?
% Do different leanings use propaganda with different topics?
% Targets of propaganda (?)
% Who are the targets of propaganda (overall)?
% Do different leanings have different targets of propaganda?
% Prediction
% Is it possible to predict the political leaning of an article from its propaganda features?
% By removing propaganda terms from articles? 



\section{Research hypotheses}
\label{sec:intro_hyp}

We have several hypotheses:

% layers of info and choice of terms
News articles are made with different layers of information: (i) facts, events (ii) interpretation/persuasion.
These layers are not separate and are very intertwined. On the word level, we have words that are strictly topical words, or are strictly persuasive words, but we also have words that represent both layers as a specific term is chosen in a multitude of synonyms to push for a certain idea.

% corroboration and choice of details
News articles are written by choosing which details to include and which ones to skip, and this may be done on purpose to influence the reader.

% propaganda and leaning
Propaganda language used by different political leanings is distinguishable

% unbalance
Current propaganda detection recognises better right-leaning propaganda

\section{Research methodology}
\label{sec:intro_method}

We introduce one factor for each chapter: Similarity, Persuasion, Leaning, Topic.

We use several methodologies:
\begin{itemize}
    \item classifier-based: F1 comparison, confusion analysis, significance tests, analysis of the features learned by the models 
    \item comparative analysis: Chapter 5 and 6 we use one observed variable across one or multiple controlled variables. We extract patterns and observations from statistics compared between the groups.
\end{itemize}


\section{Contributions}
\label{sec:intro_contributions}

Contributions:
Identification of imbalance of the main propaganda datasets
Topics where current automated propaganda detection is more problematic (TODO) 
Link between propaganda and political leaning is weak (not enough to identify leaning by just looking at the propaganda techniques) → against HYP1
Imbalance is / is-not a problem for propaganda detection
Propaganda vs populism

\section{Structure of the dissertation}
\label{sec:intro_structure}

Chapter 3: Common Ground Search
Chapter 4: Linguistic Proxies of Persuasion
Chapter 5: Perspectives and Political Sides
Chapter 6: Topics
Chapter 7: ?

\section{Publications}
\label{sec:intro_publications}

\subsection{Related to this thesis}

Towards a Cross-article Narrative Comparison of News
M Mensio, H Alani, A Willis
Proceedings of the Text2Story’20 Workshop

We present an idea for a system to perform cross-article narrative comparison.

\subsection{Other publications during the timeframe}

Mensio M., Alani H. (October 2019) News Source Credibility in the Eyes of Different Assessors. In Conference for Truth and Trust Online (TTO 2019), London, UK [Full text (ORO)] [Full text (Conference)] [Slides]

Mensio M., Alani H. (October 2019) MisinfoMe: Who’s Interacting with Misinformation? In (ISWC 2019) Posters and Demos [Poster] [Full text (ORO)] [Full text (CEUR)]


Mensio M., Bastianelli E., Tiddi I., Rizzo G. (March 2020) Mitigating Bias in Deep Nets with Knowledge Bases: the Case of Natural Language Understanding for Robots. In AAAI 2020 Spring Symposium on Combining Machine Learning with Knowledge Engineering [Full text (CEUR)]

Mensio M., Willis A., Alani H. (April 2020) Towards a Cross-article Narrative Comparison of News. In Third International Workshop on Narrative Extraction from Texts held in conjunction with the 42nd European Conference on Information Retrieval (Text2Story2020 @ ECIR2020) [Full text (ORO)] [Full text (CEUR)] [Presentation slides]

Mensio M., Alani H., Willis A. (June 2020) One Event, Different Stories. In Postgraduate Research Poster Competition 2020, The Open University [Multimedia entry (ORO)] [Poster (ORO)]

Burel G., Farrell T., Mensio M., Khare P., Alani H. (October 2020) Co-Spread of Misinformation and Fact-Checking Content during the Covid-19 Pandemic. In (SocInfo 2020) Social Informatics 2020 [Full text (ORO)] [Full text (Springer)]

Piccolo L., Blackwood A., Farrell T., Mensio M. (July 2021) Agents for Fighting Misinformation Spread on Twitter: Design Challenges. In 3rd Conference on Conversational User Interfaces (CUI 2021) [Full text (ORO)] [Full text (ACM)]

Denaux R., Mensio M., Gomez-Perez J., Alani H. (August 2021) Weaving a Semantic Web of Credibility Reviews for Explainable Misinformation Detection. In Thirtieth International Joint Conference on Artificial Intelligence (IJCAI-21) [Full text (ORO)] [Full text (IJCAI)]

Reyero Lobo P., Mensio M., Pavon-Perez A., Bayer V., Kwarteng J., Fernandez M., Daga E., Alani H. (June 2022) Estimating Ground Truth in a Low-labelled Data Regime: A Study of Racism Detection in Spanish. In (ICWSM-22) First Workshop on Novel Evaluation Approaches for Text Classification Systems on Social Media (NEATCLasS) [Full text (ICWSM)]


MARTINO MENSIO, GRÉGOIRE BUREL, TRACIE FARRELL, and HARITH ALANI. (June 2023) MisinfoMe: A Tool for Longitudinal Assessment of Twitter Accounts’ Sharing of Misinformation. In (UMAP 2023) The 31st ACM Conference On User Modeling, Adaptation And Personalization

Martino Mensio, Grégoire Burel, Youri Peskine, Raphaël Troncy, Paolo Papotti, and Harith Alani. (November 2023).
MisinfoKG - A Misinformation and Fact-Checks
Knowledge Graph. In ISWC-2023 Resource Track
