Test
\acrshort{kmi}


Motivation/Orientation: multiple narrations of the same events. Driven by multiple factors (selection criteria, point of view of author/publisher, relevance, agenda-setting, …)
This is linked with Persuasion, but more specifically propaganda. Communication with the goal to persuade. Subtle or more explicit.
What distinguishes one piece of news from the others about the same event? 

Political ideology → writer/editor → news (facts + propaganda(opinion)) → persuasion of reader

Rationale (niche): computational propaganda detection is at an early stage in the research community. Current detection is based on unbalanced datasets that particularly target right-leaning news.

Aims/Goals of this thesis: 
understand how propaganda varies across the political spectrum
Computational perspective: is propaganda detection ready to work in news independently from the political orientation/leaning of the source?

Hypotheses:
Propaganda language used by different political leanings is distinguishable
 Current propaganda detection is biased to recognise better right-leaning news



RQs:
What is the relationship between propaganda and political leaning?
Diffusion
Does propaganda exist all across political leaning?
Do existing propaganda datasets represent each leaning well?
(Why is there this discrepancy between i) and ii)?)
Term usage
What terms are used in propaganda?
Do different leanings use similar propaganda terms?
Populism
Is propaganda correlated to populism in literature?
Does the correlation show up in the data?
Topics (?)
How is propaganda spread across topics (overall)?
Do different leanings use propaganda with different topics?
Targets of propaganda (?)
Who are the targets of propaganda (overall)?
Do different leanings have different targets of propaganda?
Prediction
Is it possible to predict the political leaning of an article from its propaganda features?
By removing propaganda terms from articles? 

Contributions:
Identification of imbalance of the main propaganda datasets
Topics where current automated propaganda detection is more problematic (TODO) 
Link between propaganda and political leaning is weak (not enough to identify leaning by just looking at the propaganda techniques) → against HYP1
Inbalance is / is-not a problem for propaganda detection
Propaganda vs populism
