% intro, problem
% ORIENTATION: wider context
% News media is the way that we get informed about information every day.
% We read the news to know what happens and when we do that, not only do we receive the facts, but also the interpretation of the writer.
% that can be distinguished with different degrees of difficulty.
% This phenomenon is called \emph{framing} and different works have analysed it.
In recent years, computational studies started analysing several aspects of News Media.
Both researchers and private companies, driven by the need to understand and keep a healthy environment around public discussion, began developing models to extract and classify news articles in multiple dimensions: parallel news analysis, propaganda techniques detection, political leaning recognition and topic detection.

% RATIONALE: create a niche
While there is a lot of computational research on propaganda detection, we find that it has not been studied together with the other factors of 
%But computational studies lack a study of the relationships between these multiple factors of 
parallel news, political leaning, and topics.
It is important to study the relationship between these factors in order to have a much clearer understanding of propaganda in different conditions.
% With this PhD, we want to tackle instances of framing that are difficult to recognise.
% These can manifest subtly through word choices and selection of what to include or exclude from the narrative line.


% AIM: purpose of research
This PhD aims to analyse the relationship between propaganda and parallel news, political leaning and topics. We want to understand how propaganda changes across political leaning and topics.


% METHOD: methodology and theoretical framework 
We use two methodologies in our work:
(i) comparative analysis: we study one observed dimension across one or more partitioning variables;
(ii) classifier-based: we combine different variables as features for a classifier that needs to predict one additional variable.
With these two methodologies, we analyse how propaganda changes across leanings and topics, and we see what is the effect of using propaganda and topics to predict the political leaning of news articles. 


% contributions
% FINDINGS: outline of the findings
Our work has three main contributions.
(i) We quantify the relationships between propaganda and similarities, political leaning and topics. We discover that for specific topics, news sources use very different propaganda terms across the political spectrum.
(ii) We improve the F1 of a political leaning classifier using the features of propaganda and topic.
(iii) We discover an important imbalance in the datasets used for propaganda detection: most of the articles come from right-leaning sources.
% We present a method that could help both human readers and automated approaches to find more easily when such framing techniques occur in the news.
% We want to provide an analysis of articles, by exploiting the differences in how information is presented by different sources.
% We can then analyse how different sources in the media landscape are using framing in different ways.  


% INTERPRETATION: general significance of the findings 
We are convinced that the results found in this work can be useful both for computational researchers and users reading the news.
For the first group, we emphasise the need to exploit the relationships discovered to improve models (using mixed features) and datasets (making them more representative of the news as a whole).
For the second group, our work can be applied to build tools to empower the reasoning of users, by letting them compare multiple articles on the same topic and highlighting the propaganda techniques used.
