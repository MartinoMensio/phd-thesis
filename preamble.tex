%%%%%%%%%%%%%%%%%%%%%%%%%%%%%%%%%%%%%%%%%%%%%%%%%%%%%%%%%%%%%\
%%%% Use packages (add here your additional packages)

%\usepackage{amsfonts}

%%% For figures
\usepackage{graphicx}
%\usepackage{subfig,rotate}

%%% for comments
\usepackage{verbatim}

%%% For tables
\usepackage{multirow}
% Longtable lets you have tables that span multiple pages.
\usepackage{longtable}

% Booktabs produces far nicer tables than the standard LaTeX tables.
%   see: http://en.wikibooks.org/wiki/LaTeX/Tables
\usepackage{booktabs}

%set parameters for longtable:
% default caption width is 4in for longtable, but wider for normal tables
\setlength{\LTcapwidth}{\textwidth}



%%%% End packages
%%%%%%%%%%%%%%%%%%%%%%%%%%%%%%%%%%%%%%%%%%%%%%%%%%%%%%%%%%%%%\


%%%%%%%%%%%%%%%%%%%%%%%%%%%%%%%%%%%%%%%%%%%%%%%%%%%%%%%%%%%%%\
%%%% other stuff (better to not change)

\newcommand{\proquestmode}{}
% I prefer proquestmode to be off for electronic copies for normal use, since the colored links are less distracting. However when printed in black and white, the colored links are difficult to read. 


% Alter some LaTeX defaults for better treatment of figures:
% See p.105 of "TeX Unbound" for suggested values.
% See pp. 199-200 of Lamport's "LaTeX" book for details.
%   General parameters, for ALL pages:
\renewcommand{\topfraction}{0.85}	% max fraction of floats at top
\renewcommand{\bottomfraction}{0.6}	% max fraction of floats at bottom
%   Parameters for TEXT pages (not float pages):
\setcounter{topnumber}{2}
\setcounter{bottomnumber}{2}
\setcounter{totalnumber}{4}     % 2 may work better
\setcounter{dbltopnumber}{2}    % for 2-column pages
\renewcommand{\dbltopfraction}{0.66}	% fit big float above 2-col. text
\renewcommand{\textfraction}{0.15}	% allow minimal text w. figs
%   Parameters for FLOAT pages (not text pages):
\renewcommand{\floatpagefraction}{0.66}	% require fuller float pages
% N.B.: floatpagefraction MUST be less than topfraction !!
\renewcommand{\dblfloatpagefraction}{0.66}	% require fuller float pages

% The documentclass already sets parameters to make a high penalty for widows and orphans. 


%%%%%%%%%%%%%%%%%%%%%%%%%%%%%%%%%%%%%%%%%%%%%%%%%%%%%%%%%%
%%% Printed vs. online formatting
\ifdefined\printmode

% Printed copy
% url package understands urls (with proper line-breaks) without hyperlinking them
\usepackage{url}


\else

\ifdefined\proquestmode
%ProQuest copy -- http://www.princeton.edu/~mudd/thesis/Submissionguide.pdf

% ProQuest requires a double spaced version (set previously). They will take an electronic copy, so we want links in the pdf, but also copies may be printed or made into microfilm in black and white, so we want outlined links instead of colored links.
\usepackage{hyperref}
\hypersetup{bookmarksnumbered}

% copy the already-set title and author to use in the pdf properties
\makeatletter
\hypersetup{pdftitle=\@title,pdfauthor=\@author}
\makeatother

\else
% Online copy

% adds internal linked references, pdf bookmarks, etc

% turn all references and citations into hyperlinks:
%  -- not for printed copies
% -- automatically includes url package
% options:
%   colorlinks makes links by coloring the text instead of putting a rectangle around the text.
\usepackage{hyperref}
\hypersetup{colorlinks,bookmarksnumbered}

% copy the already-set title and author to use in the pdf properties
\makeatletter
\hypersetup{pdftitle=\@title,pdfauthor=\@author}
\makeatother

% make the page number rather than the text be the link for ToC entries
%\hypersetup{linktocpage}
\fi % proquest or online formatting
\fi % printed or online formatting


%%%%%%%%%%%%%%%%%%%%%%%%%%%%%%%%%%%%%%%%%%%%%%%%%%%%%%%%%%%%%\
%%%% Define commands

% Define any custom commands that you want to use.
% For example, highlight notes for future edits to the thesis
%\newcommand{\todo}[1]{\textbf{\emph{TODO:}#1}}


% create an environment that will indent text
% see: http://latex.computersci.org/Reference/ListEnvironments
% 	\raggedright makes them left aligned instead of justified
\newenvironment{indenttext}{
    \begin{list}{}{ \itemsep 0in \itemindent 0in
    \labelsep 0in \labelwidth 0in
    \listparindent 0in
    \topsep 0in \partopsep 0in \parskip 0in \parsep 0in
    \leftmargin 1em \rightmargin 0in
    \raggedright
    }
    \item
  }
  {\end{list}}

% another environment that's an indented list, with no spaces between items -- if we want multiple items/lines. Useful in tables. Use \item inside the environment.
% 	\raggedright makes them left aligned instead of justified
\newenvironment{indentlist}{
    \begin{list}{}{ \itemsep 0in \itemindent 0in
    \labelsep 0in \labelwidth 0in
    \listparindent 0in
    \topsep 0in \partopsep 0in \parskip 0in \parsep 0in
    \leftmargin 1em \rightmargin 0in
    \raggedright
    }

  }
  {\end{list}}



%%%%%%%%%%%%%%%%%%%%%%%%%%%%%%%%%%%%%%%%%%%%%%%%%%%%%%%%%%%%%\
%%%% Front-matter


% For early drafts, you may want to disable some of the frontmatter. Simply change this to "\ifodd 1" to do so.
\ifodd 0
% front-matter disabled while writing chapters
\renewcommand{\maketitlepage}{}
\renewcommand*{\makecopyrightpage}{}
\renewcommand*{\makeabstract}{}

% you can just skip the \acknowledgements and \dedication commands to leave out these sections.

\else


\abstract{
% Abstract can be any length, but should be max 350 words for a Dissertation for ProQuest's print indicies (150 words for a Master's Thesis) or it will be truncated for those uses.
% intro, problem
% ORIENTATION: wider context
% News media is the way that we get informed about information every day.
% We read the news to know what happens and when we do that, not only do we receive the facts, but also the interpretation of the writer.
% that can be distinguished with different degrees of difficulty.
% This phenomenon is called \emph{framing} and different works have analysed it.
In recent years, computational studies started analysing several aspects of News Media.
% Both researchers and private companies, driven by the need to understand and keep a healthy environment around public discussion, began
The research community began
developing models to extract and classify news articles in multiple dimensions: parallel news reports analysis, persuasion techniques detection, political leaning recognition and topic detection.

% RATIONALE: create a niche
While there is a lot of computational research on persuasion detection, we find that it has not been studied together with the other factors of 
%But computational studies lack a study of the relationships between these multiple factors of 
parallel news reports, political leaning, and topics.
% It is important to study the relationship between these factors in order to have a much clearer understanding of propaganda in different conditions.
% With this PhD, we want to tackle instances of framing that are difficult to recognise.
% These can manifest subtly through word choices and selection of what to include or exclude from the narrative line.
%
%
% AIM: purpose of research
% This PhD aims to analyse the relationship between propaganda and parallel news reports, political leaning and topics. We want to understand how propaganda changes across political leaning and topics.
This PhD analyses the use of language in parallel news articles to understand how persuasion changes across political leaning and topics.


% METHOD: methodology and theoretical framework 
We use two methodologies in our work:
(i) comparative analysis: we study one observed dimension across one or more partitioning variables;
(ii) classifier-based: we combine different variables as features for a classifier that needs to predict one additional variable.
With these two methodologies, we analyse how persuasion changes across leanings and topics, and we see what is the effect of using persuasion and topics to predict the political leaning of news articles. 


% contributions
% FINDINGS: outline of the findings
Our work has three main contributions.
(i) We improve the F1 of a political leaning classifier using the features of propaganda and topic.
(ii)
%We quantify the relationships between propaganda and similarities, political leaning and topics. 
We discover that for specific topics, news sources use very different propaganda terms across the political spectrum, while still using the same techniques.
% The same persuasion techniques are used for each topic, even when the political leaning of the article is different.
(iii) We discover an imbalance in the standard datasets for propaganda detection, which raises questions about the generality of results in the literature.
% We present a method that could help both human readers and automated approaches to find more easily when such framing techniques occur in the news.
% We want to provide an analysis of articles, by exploiting the differences in how information is presented by different sources.
% We can then analyse how different sources in the media landscape are using framing in different ways.  


% INTERPRETATION: general significance of the findings 
We believe that the results of this work can be useful both for computational researchers and users reading the news.
For the first group, we emphasise the need to exploit the relationships discovered to improve models (using mixed features) and datasets (making them more representative of the news as a whole).
% For the second group, our work can be applied to build tools to empower the reasoning of users, by letting them compare multiple articles on the same topic and highlighting the techniques used.
For the second group, our work can be applied to build tools to help users compare multiple articles on the same topic, to highlight any techniques used.

}

\acknowledgements{
%I would like to thank...
Above all, I want to express my gratitude to my supervisors, Prof. Harith Alani and Dr. Alistair Willis, for their unwavering support and guidance throughout my research. They were always available to discuss any problems or inquiries I had and provided valuable feedback.
I would like to give a special thanks to Prof. Harith Alani for his thought-provoking questions that helped me stay on track with my research. His talent for simplifying complex ideas into understandable concepts demonstrated that every problem can be solved with the proper approach and clear explanations.
I am grateful for the technical assistance provided by Dr. Alistair Willis, which enabled me to effectively frame and focus the most intricate section of my thesis. Through his invaluable support, I gained valuable insight into designing strong experiments and critically analysing results.

% guidance showed me the importance of good writing in research. Hopefully, his ability to spot the smallest typos and grammatical errors made me a better writer.

I would like to express my gratitude to my friends and colleagues at the Knowledge Media Institute (KMi), Computing \& Communications (C\&C) and generally in the Open University family, for their support throughout my studies. Their assistance has been invaluable to me. Paula, Angel, Angelo, Lucas, Lara, Alba, Agnese, Gianluca, Perla, Tommy, Pedro, Thais, Patrizia, Julian, Miriam, Cecilia and Paco: you all contributed to making life in Milton Keynes better.

And thanks to the two musical bands that accompanied me during this 3-year period, \emph{KMinstruments} and \emph{I Gabbiani}, that provided a good and motivating distraction and made the time more enjoyable. 

Finally, I would like to thank my mum, dad, brother, and sister for supporting me even in the distance over all these years, and spurring me to continue.% and not give up.

Last but not least, I would like to thank Giorgia.
%for her unofficial supervising abilities and for standing on my side.
%and motivation to push me all the times I lost my energy.
Your support has been an inexhaustible source of comfort and motivation for me in all the tough times. Your positive energy and determined encouragement always kept me going, even when things seemed impossible. Your ability to put things into perspective, and focus on the finish line, helped me remain determined and focused on my goals. I am truly grateful for your persistent support and will always cherish it.

%Her ability to deal with a PhD researcher, stressed and anxious about the experiments failing and this never-ending thesis, cannot be expressed in words.

}

\dedication{
Dediction
}

\fi  % disable frontmatter


%%%%%%%%%%%%%%%%%%%%%%%%%%%%%%%%%%%%%%%%%%%%%%%%%%%%%%%%%%%%%\
%%%% Hide some chapters

%%% If you want to produce a pdf that includes only certain chapters, specify them with includeonly, in addition to including all chapters below.
%\includeonly{ch-intro/chapter-intro}
%%% You can also specify multiple chapters.
%\includeonly{ch-intro/chapter-intro,ch-usage/chapter-usage}
%\includeonly{chap1,chap2,chap3}

